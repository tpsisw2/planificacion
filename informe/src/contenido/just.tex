\section{Justificación}

Para encarar este encargo, se decidió hacer una iteración de 6 semanas  (considerando que desde la fecha de inicio hasta la fecha de entrega hay 6 semanas). Para cada integrante, considerando que cada uno trabaja y tiene que ir a la facultad, estimamos un total de 20 horas por semana para cada uno que podría dedicarle al proyecto. Por lo tanto, considerando que somos 4 integrantes, esto nos da un total de aproximadamente 80 horas semanales osea, 480 horas para la iteración. Habiendo decidido esto, teniendo el backlog, hubo que decidir qué tareas se iban a tomar en qué sprint. Para hacer esto, tuvimos en cuenta que había funcionalidades que necesitaba estar desplegada necesariamente  para noviembre (poder postear contenido multimedia y loguearse al sistema), por lo tanto, esto era lo más prioritario, ya que aún si se retrasaran un poco tendríamos otra iteración para poder terminarlas. De esta manera, tratamos de minimizar el riesgo de que no podamos cumplir con las funcionalidad esa tiempo y que pueda caerse el proyecto.

En todos los casos de uso que corresponden a la primera iteración, se realizó un desgloce de los mismos en tareas, para poder estimar mejor los tiempos que insume cada uno y además poder analizar las dependencias entre ellas, de esta manera se puede observar mejor cómo encarar la itereación y qué tareas es necesario tomar antes porque hay otras que requieren su finalización. Sin embargo, hay que tener en cuenta también es que a la hora de definir el backlog, y consecuentemente las iteraciones, existen tareas que no pertenecen a ningún caso de uso, ya que no representan ninguna interacción de un usuario con el sistema, pero que las agregamos porque es necesario estimarlas ya que van a insumir tiempo; de las 480 horas disponibles dedicamos 20 a la realización del informe y 30 al planteo de la arquitectura quedando 430 horas disponibles en la iteración para realizar las tareas asociadas a los casos de uso seleccionados para esta etapa.

